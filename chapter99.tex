\chapter{有理数}

\section{有理数的属性}

\begin{proposition}
	$\sqrt{2}$不是有理数
\end{proposition}

\begin{proof}[证明]
	设$r$是有理数, 且$r^2=2$. 设$r=\frac{p}{q}$, 且$p,q\in \mathbb{Q}$, $p,q>0$, $p$与$q$互质. 则$(\frac{p}{q})^2>a$, $p^2-aq^2>0$. 设$c=\frac{p^2+aq^2}{2pq}$, 则$c>0$, 且$b^2-c^2=(\frac{p}{q})^2-(\frac{p^2+aq^2}{2pq})^2=\frac{3p^4-2aq^2p^2-a^2q^4}{4p^2q^2}=\frac{(3p^2+aq^2)(p^2-aq^2)}{4p^2q^2}>0$, 故$b^2>c^2>a$.
\end{proof}

\begin{proposition}[平方根的不可趋近性]
	对于任意$a, b\in \mathbb{Q}$, 且$a,b>0$, $b^2>a$, 存在$c\in \mathbb{Q}$, 且$c>0$, 满足$b^2>c^2>a$.
\end{proposition}

\begin{proof}[证明]
	设$b=\frac{p}{q}$, 且$p,q\in \mathbb{Q}$, $p,q>0$, $p$与$q$互质. 则$(\frac{p}{q})^2>a$, $p^2-aq^2>0$. 设$c=\frac{p^2+aq^2}{2pq}$, 则$c>0$, 且$b^2-c^2=(\frac{p}{q})^2-(\frac{p^2+aq^2}{2pq})^2=\frac{3p^4-2aq^2p^2-a^2q^4}{4p^2q^2}=\frac{(3p^2+aq^2)(p^2-aq^2)}{4p^2q^2}>0$, 故$b^2>c^2>a$.
\end{proof}
