% !TeX encoding = utf8

\documentclass[UTF8]{ctexbook}
\usepackage{graphicx}
\usepackage{amssymb}

\title{数理逻辑笔记}
\author{张东升}
\date{}

\newtheorem{theorem}{定理}

\begin{document}
\maketitle
\thispagestyle{empty}

\tableofcontents
\thispagestyle{empty}
\pagenumbering{Roman}

\chapter{前言}
\thispagestyle{empty}
\begin{quote}
	\zihao{-5}\kaishu 长远而言,是观念,因而也正是传播新观念的人,主宰着历史发展的进程。\footnote{Friedrich Hayek}
\end{quote}
记录学习过程中的点点滴滴。

\begin{figure}[ht]
	\includegraphics[width=10cm]{Walle7.jpg}
\end{figure}

\chapter{公理化形式系统}
\pagenumbering{arabic}
\section{形式化推理}
\subsection{形式化公理}
\begin{enumerate}
	\item $A\rightarrow (B\rightarrow A)$
	\item $(A\rightarrow (B\rightarrow C))\rightarrow ((A\rightarrow B)\rightarrow (A\rightarrow C))$
	\item $(\neg A\rightarrow \neg B)\rightarrow (B\rightarrow A)$
\end{enumerate}

\subsection{推理规则}
$A,A\rightarrow B\vdash A$

\section{定理}
\subsection{$A\rightarrow A$}
\begin{enumerate}
	\item $A\rightarrow ((A\rightarrow A)\rightarrow A)$
	\item $(A\rightarrow ((A\rightarrow A)\rightarrow A))\rightarrow ((A\rightarrow (A\rightarrow A))\rightarrow (A\rightarrow A))$
	\item $(A\rightarrow (A\rightarrow A))\rightarrow (A\rightarrow A)$
	\item $A\rightarrow (A\rightarrow A)$
	\item $A\rightarrow A$
\end{enumerate}

\subsection{$A\rightarrow (\neg A\rightarrow B)$}
\begin{enumerate}
	\item $\neg A\rightarrow ((\neg B\rightarrow \neg A)\rightarrow (A\rightarrow B))$
	\item $(\neg A\rightarrow ((\neg B\rightarrow \neg A)\rightarrow (A\rightarrow B)))\rightarrow ((\neg A\rightarrow (\neg B\rightarrow \neg A))\rightarrow (\neg A \rightarrow (A\rightarrow B))$
	\item $(\neg A\rightarrow (\neg B\rightarrow \neg A))\rightarrow (\neg A \rightarrow (A\rightarrow B)$
	\item $\neg A\rightarrow (\neg B\rightarrow \neg A)$
	\item $\neg A\rightarrow (A\rightarrow B)$
\end{enumerate}

\begin{theorem}
	演绎定理:若$\Gamma\cup \{A\}\vdash B$,则$\Gamma \vdash (A\rightarrow B)$
\end{theorem}

\chapter{集合论}
\section{有序对}
\subsection{Norbert Wiener有序对}
	$(a, b):=\{ \{ \varnothing , \{ a\} \} , \{ \{ b\} \} \}$

\bibliography{math}

\end{document}
